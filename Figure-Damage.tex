\documentclass[border=10pt]{standalone}
\usepackage{pgfplots}
\pgfplotsset{width=7cm, compat=1.10}
\usepgfplotslibrary{fillbetween}
\pgfmathdeclarefunction{poly}{0}{\pgfmathparse{1-x}}

\begin{document}
\begin{tikzpicture}
  \begin{axis}[
    axis x line=middle,
    axis y line=middle,
    xtick       = {0.5,1},
    xticklabels = {$\mathbf{\omega}_n$,$\mathbf{\omega}_c$},
    ytick       = {0.5,1},
    yticklabels = {$\mathbf{f}_c$,$\mathbf{f}_t$},
    samples     = 160,
    domain      = 0:1,
    xmin = 0, xmax = 1.2,
    ymin = 0, ymax = 1.2,
    ]
    \node[above right, black] at (15,15) {$\mathbf{G}_f$};
    \addplot[name path=poly, black, thick, mark=none, ] {poly};
    \addplot[name path=line, black, no markers, line width=0.1pt] {0};
    \addplot fill between[ 
    of = poly and line, 
    split, % calculate segments
    every even segment/.style  = {gray!60}
    ];
    \draw[dashed] (axis cs:0,0.5) -| (axis cs:0.5,0);     
  \end{axis} 
\end{tikzpicture}
\end{document}

% \documentclass[border=10pt]{standalone}
% \usepackage{pgfplots}
% \usepackage{tikz}
% \begin{document}
% \begin{tikzpicture}
%   % Draw axes
%   \draw [<->,thick] (0,1.2) node (yaxis) [above] {$y$}
%   |- (1.2,0) node (xaxis) [right] {$x$};
%   % Draw two intersecting lines
%   % \draw (0,0) coordinate (a_1) -- (2,1.8) coordinate (a_2);
%   \draw (0,1) coordinate (b_1) -- (1,0) coordinate (b_2);
%   % % Calculate the intersection of the lines a_1 -- a_2 and b_1 -- b_2
%   % % and store the coordinate in c.
%   % \coordinate (c) at (intersection of a_1--a_2 and b_1--b_2);
%   % % Draw lines indicating intersection with y and x axis. Here we use
%   % % the perpendicular coordinate system
%   \coordinate (c) at (0.5,0.5);  
%   \draw[dashed] (yaxis |- c) node[left] {$y'$}
%   -| (xaxis -| c) node[below] {$x'$};
%   % % Draw a dot to indicate intersection point
%   % \fill[red] (c) circle (2pt);
% \end{tikzpicture}
% \end{document}