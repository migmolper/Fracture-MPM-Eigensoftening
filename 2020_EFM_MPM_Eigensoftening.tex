\documentclass[preprint,12pt,a4paper]{elsarticle}

\usepackage{lineno,hyperref}
\usepackage{float}
\usepackage{subfig}
\usepackage{color}
\usepackage{soul}

\usepackage{algcompatible}
\usepackage{algorithm}
\usepackage{algorithmic}
\usepackage{algpseudocode}

\usepackage{cancel}

\usepackage{amsmath,amsthm,amssymb}

\newcommand{\vec}[1]{
  \ensuremath{\mathbf{{#1}}}
}
\newcommand{\tens}[1]{
  \ensuremath{\mathbf{{#1}}}
}
\newcommand{\Matrix}[1]{
  \ensuremath{\mathbf{{#1}}}
}
\newcommand{\Vector}[1]{
  \ensuremath{\mathbf{{#1}}}
}
% Divergence
\newcommand{\Div}[1]{
  \ensuremath{div({#1})}
}
% Gradient
\newcommand\Grad[1]{grad({#1})}
\newcommand\GradS[1]{grad^s({#1})}
\newcommand\GradT[1]{grad^T({#1})}
% Partial derivative
\newcommand{\Deriv}[3][]{
  \ensuremath{\frac{\partial^{#1}{#2}}{ \partial {#3}^{#1} }}
}
% Integral
\newcommand{\Integral}[2]{
  \IfStrEqCase{#1}{
    {2}{\ensuremath{\int_{\varGamma_d}{#2}\ d\varGamma}}
    {3}{\ensuremath{\int_{\varOmega}{#2}\ d\varOmega}}
  }
}

\modulolinenumbers[5]

\journal{Engineering Fracture Mechanics}

%%%%%%%%%%%%%%%%%%%%%%%
%% Elsevier bibliography styles
%%%%%%%%%%%%%%%%%%%%%%%
%% To change the style, put a % in front of the second line of the current style and
%% remove the % from the second line of the style you would like to use.
%%%%%%%%%%%%%%%%%%%%%%%

%% Numbered
%\bibliographystyle{model1-num-names}

%% Numbered without titles
%\bibliographystyle{model1a-num-names}

%% Harvard
%\bibliographystyle{model2-names.bst}\biboptions{authoryear}

%% Vancouver numbered
%\usepackage{numcompress}\bibliographystyle{model3-num-names}

%% Vancouver name/year
%\usepackage{numcompress}\bibliographystyle{model4-names}\biboptions{authoryear}

%% APA style
%\bibliographystyle{model5-names}\biboptions{authoryear}

%% AMA style
%\usepackage{numcompress}\bibliographystyle{model6-num-names}

%% `Elsevier LaTeX' style
\bibliographystyle{elsarticle-num}
%%%%%%%%%%%%%%%%%%%%%%%

\begin{document}

\begin{frontmatter}

\title{LME-MPM applied to quasi-brittle fracture.}

%% Group authors per affiliation:
\author{
Miguel Molinos$^a$\footnote{Corresponding author: m.molinos@outlook.es},
and Pedro Navas$^a$\footnote{Corresponding author: p.navas@upm.es}
 }
 \address{
 $^a$ ETSI Caminos, Canales y Puertos, Universidad Polit\'ectnica de Madrid.\\ c. Prof. Aranguren 3, 28040 Madrid, Spain
}

\begin{abstract}

  The objective of this work is to introduce an alternative
  technique to address the fracture process of brittle and
  quasi-brittle materials under the material point method (MPM)
  framework. With this purpose the eigensoftening algorithm, developed
  originally for the optimal transportation meshfree (OTM)
  approximation scheme, is extended to the MPM with the aim of present
  a suitable alternative to the existing fracture algorithms developed
  for the MPM. The good fitting in the predictions made by the
  eigensoftening algorithm against both analytical and experimental
  results proofs the well performance of the method under challenging loads.

\end{abstract}

\begin{keyword}
Quasi brittle fracture \sep Local-\textit{max-ent} approximation \sep
Material Point Method \sep Solid Dynamics
\end{keyword}

\end{frontmatter}

\linenumbers

%%%%%%%%%%%%%%%%%%%%%%%%%%%%%%%%%%%%%%%%%%%%%%%%%%%%%%%%%%%%%%%%%%%%%%%%%
\section{Introduction}
\label{sec:1}

The simulation of fracture propagation in a more accurate and
effective way can be considered as one of the original drivers for
developing novel spatial discretization methods such as meshfree
methods like the material point method (MPM). Presence of cracks are a violation of the
continuity requirement of the finite element approach (MPM).

The MPM does not suffer from the above difficulties. Discontinuities
can be described in two ways. One is to abandon the single-valued
velocity field property near the crack by using two or more background
meshes, and the other is to use failed material points to
approximately describe the crack.

In the first approach,



%%%%%%%%%%%%%%%%%%%%%%%%%%%%%%%%%%%%%%%%%%%%%%%%%%%%%%%%%%%%%%%%%%%%%%%%%
\section{The meshfree methodology}
\label{sec:2}

The popularity of MPM has increase notoriously during the recent years
due to its ability to deal with large strain problems without mesh
distorsion issues inherent to mesh based methods like FEM, see
Zdzislaw \cite{Wieckowski2004}. However, in the simulations made with
the original MPM, there are numerical noises when particles crossing
the cell boundaries. 

\subsection{The Material Point Method}
\label{sec:2.1}

For the spatial discretization, two sets of points are introduced in
the MPM. First, the nodes, this points are considered fixed in the space and are in charge of
computing all the kinematic fields such forces $f_I$, accelerations $a_I$ and
velocities $v_I$. And second the material points or particles. They are in
charge of the discretization of the continuum, and store the local state ($\sigma_p, \varepsilon_p$).

\subsection{Spatial discretization : Local-\textit{max-ent} approximants}
\label{sec:2.2}

Local maximum-entropy (or local \textit{max-ent}) approximation scheme
was introduced by Arroyo \& Ortiz (2006)\cite{Arroyo2006} as a bridge
between finite elements and meshfree methods. The key idea of the 
shape functions is to interpret the nodal of a shape function $N_I$ as
a probability. This allow us to introduce two important limits:
the principle of maximum-entropy (\textit{max-ent}) statistical
inference stated by \cite{Jaynes1957}, and the Delaunay triangulation
which ensures the minimal width of the shape function. To reach to a compromise between two competing objectives, a Pareto set is defined as, 
\begin{align*}
  \label{eq:LME-scheme-pareto-set}
  \text{(LME)}_{\beta} \hspace{0.15cm} &\text{For fixed} \hspace{0.15cm}
  \vec{x} \hspace{0.15cm} \text{minimise} \hspace{0.15cm} f_{\beta}(\vec{x}_p, N_I) = \beta U(\vec{x}_p,N_I) - H(N_I) \\
  &\text{subject to}\
  \begin{cases}
    N_I \ge 0, \hspace{0.15cm} \text{I=1, ..., n} \\[1em]   
    \sum\limits_{I=1}^{N_n}{N_I} = 1 \\[1em]   
    \sum\limits_{I=1}^{N_n}{N_I \vec{x}_I} = \vec{x} \\
  \end{cases}
\end{align*}
where $H(N_I)$ is the entropy of the system of nodes following the
definition given by Shannon (1948) \cite{Shannon1948}, and $U(\vec{x}_p,N_I) =
\sum_I N_I |\vec{x}_p - \vec{x}_I |^2$ a magnitude of the shape
function width. The regularization o \textit{thermalization} parameter
between the two criterion $\beta$ has Pareto optimal values in the range
$(0,\infty)$. The unique solution of the local max-ent problem
(LME)$_\beta$ is:
\begin{equation}
  \label{eq:LME-p}
N_I^*(\vec{x})=\frac{\exp\left[ -\beta \; |\vec{x}-\vec{x}_I|^2 +
    \vec{\lambda}^* \cdot (\vec{x}-\vec{x}_I) \right] } {Z(\vec{x},\vec{\lambda}^*(\vec{x}))}
\end{equation}
where $Z(\vec{x},\vec{\lambda}^*(\vec{x}))$ is the \textit{partition
  function} defined as,
\begin{equation}
  \label{eq:LME-Z}
Z(\vec{x}, {\vec{\lambda}}) = \sum_{I=1}^{N_n}{ \exp \left[ -\beta \; |\vec{x}-\vec{x}_I|^2 + \vec{\lambda} \cdot (\vec{x}-\vec{x}_I)  \right]}
\end{equation}
and evaluated in the unique minimiser $\vec{\lambda}^*(\vec{x})$ for
the function $\log
Z(\vec{x}, \vec{\lambda})$. The traditional way to obtain such a
minimiser is using (\ref{eq:LME-J}) to calculate small increments of $\partial\vec{\lambda}$ in a
Newton-Raphson approach. Where $\tens{J}$ is the Hessian matrix, defined by:
\begin{eqnarray}
  \label{eq:LME-J} 
  \tens{J}(\vec{x}, \vec{\lambda},\beta) &\equiv& \frac{\partial
                                                  \vec{r}}{\partial \vec{\lambda}}\\
  \label{eq:LME-r}
  \vec{r}(\vec{x},\vec{\lambda},\beta) &\equiv& \frac{\partial \log{ Z(   \vec{x},\vec{\lambda}})}{\partial \vec{\lambda}}  = \sum_I^{N_n} p_I(\vec{x},\vec{\lambda},\beta) \, (\vec{x} - \vec{x}_I)
\end{eqnarray}
In order to obtain the first derivatives of the shape function~$\nabla
N^*_I$, can be computed as,
\begin{equation}
  \label{eq:LME-grad-p}
\nabla N^*_I = N^*_I  \, \left(\nabla f^*_I-\sum_J^{N_n} N^*_J \, \nabla f^*_J\right)
\end{equation}
where
\begin{equation}
  \label{eq:LME-f}
f^*_I(\vec{x},  \vec{\lambda},\beta)=-\beta \, |\vec{x}-\vec{x}_I|^2 + \vec{\lambda}^*  \,  (\vec{x}-\vec{x}_I)
\end{equation}
Employing the chain rule over \eqref{eq:LME-grad-p}, rearranging and considering $\beta$ as a
constant, Arroyo and Ortiz~\cite{Arroyo2006} obtained the following
expression for the gradient of the shape function.
\begin{eqnarray}
  \label{eq:LME-gradp} 
\nabla N^*_I &=& -N^*_I \,  (\tens{J}^*)^{-1} \,  (\vec{x} - \vec{x}_I)
\end{eqnarray}

The regularization parameter $\beta$ of LME shape functions may be
controlled by adjusting a dimensionless parameter, $\gamma=\beta h^2$
\cite{Arroyo2006}, where $h$ is defined as a measure of the nodal
spacing. Since  $N_I$ is defined in the entire domain, in practice,
the shape function decay $\exp(-\beta \vec{r} )$ is truncated  by  a
given tolerance, 10$^{-6}$, for example,  would ensure a reasonable range of
neighbours, see \cite{Arroyo2006} for details. This tolerance defines
the limit values of the influence radius and is used thereafter to
find the neighbour nodes of a given integration point.

\subsection{Temporal discretization}
\label{sec:2.3}

This research is devoted to capture the challenging process of
fracture during high velocity impacts. To capture the presence of
elastic shock waves triggered by the fracture process. Here we adopt
a explicit predictor-corrector time integration
scheme. It is based in the Newmark a-form 
$\gamma = 0.5$ and $\beta = 0$ which is the central difference
explicit. In a first stage, the nodal velocity predictor is computed
following \eqref{eq:Predictor-velocity}, 
\begin{equation}
  \label{eq:Predictor-velocity}
  \vec{\tilde{v}}_I^{k+1} = \frac{ N_{Ip}^{k} m_p (\vec{v}_p^k + (1 - \gamma)\ \Delta t\ \vec{a}_p^k)}{m_I}
\end{equation}
This way of computing the nodal predictor is both numerically stable
and minimize the computational effort. Once nodal velocity are
obtained, the essential boundary conditions are imposed. And in the
following, the ``classic'' MPM algorithm continues to reach to the
equilibrium equation \eqref{eq:particle_balance_forces3}. Here we
continue with the \textit{corrector} stage, due to the fact that we
already have nodal velocity, this step is computed in the same way as
in FEM,
\begin{equation}
  \label{eq:Corrector-velocity}
  \vec{v}_{I}^{k+1} = \vec{v}_{I}^{pred} + \gamma\ \Delta t\ \frac{\vec{f}_{I}^{k+1}}{\tens{m}_I^{k+1}}
\end{equation}
Finally updated particle kinetics are computed using nodal values as,
\begin{align}
  \label{eq:Update-lagrangian-pce}
        &\vec{a}_p^{k+1} = \frac{N_{Ip}^k\vec{f}_{I}^{k}}{\tens{m}_I^k}\\
      &\vec{v}_p^{k+1} = \vec{v}_p^n + \Delta t\
        \frac{N_{Ip}^k\
        \vec{f}_{I}^{k}}{\tens{m}_I^k}\\
      &\vec{x}_p^{k+1} = \vec{x}_p^n + \Delta t\
         N_{Ip}^k\ \vec{v}_{I}^{k} +
        \frac{1}{2}\Delta t^2\ \frac{N_{Ip}^k\
        \vec{f}_{I}^{k}}{\tens{m}_I^k} 
\end{align}
The complete pseudo-algorithm it is summarized in \ref{sec:expl-pred-corr}.

\subsection{Fracture modelling approach}
\label{sec:2.4}

Within the context of MPM formulation, fracture can be modelled by
failing particles according to a suitable criterion. When material
points are failed, they are assumed to have a null stress
tensor. Navas {\it et al.} (2017)\cite{Navas_2017_ES} developed a
eigensoftening algorithm as an extension for quasi-brittle materials
of the eigenerosion proposed by Pandolfi \& Ortiz
(2012)\cite{Pandolfi_2012} for fracture of brittle materials. A
comparison between both in \cite{Navas_2017_ES} shows that
the eigenerosion algorithm significantly overestimates the tensile
stress and the strain peaks, while it captures the forces and crack
patterns accurately. On the other hand eigensoftening algorithm agree
very well with experimental results in all the aspects. Furthermore,
this algorithm has also proof its accuracy for complex fracture
patters such the present in fiber reinforces concrete (FRC),
\cite{Navas_2018_ES}.\\

\begin{align}
  \label{eq:energy-release-EE}
&G_p^{k+1} = \frac{C_{\epsilon}}{m_p^{k+1}}  \sum_{x_q^{k+1} \in
  B_{\epsilon}(x_p^{k+1})} m_q W_q^{k+1}\\
&m_p^{k+1} =  \sum_{x_q^{k+1} \in
  B_{\epsilon}(x_p^{k+1})} m_q  
\end{align}
where $B_{\epsilon}(x_p^{k+1})$ is the sphere of radius $\epsilon$
centered at $x_p^{k+1}$ known as the $\epsilon$-neighborhood of the
material point, $m_p^{k+1}$ is the mass of the neighborhood at step
$k+1$, $W_q^{k+1}$ is the current free-energy density per unit mass as
the material point $x_q^{k+1}$ of the neighborhood, finally
$C_{\epsilon}$ is a normalizing constant. This configuration
conveniently is sketched in Figure \ref{fig:Failed-particles}.
%%%%%%%%%%%
\begin{figure}
  \centering
  \includegraphics[width=0.5\textwidth]{Figures/Particle-failed}
  \caption{Scheme of a linear cohesive law, where the shaded area is
    $G_f$, $f_t$ is the tensile strength, and $w_c$ is the critical
    opening displacement.}
  \label{fig:Failed-particles}
\end{figure}
%%%%%%%%%%%
The material point fails when $G_p^{k+1}$ surpasses a critical energy
release rate that measures the material-specific energy, $G_F$. The
convergence of this approach has been analyzed by Schmidt {\it et al.}
(2009)\cite{Schmidt_2009}, who proof that it converges to the Griffith
fracture when discretization size tends to zero. It is necessary to
point out that when a material point overpass the critical energy, its
contribution to the internal forces vector  is set to zero, but its
contribution to the mass matrix is maintained. The mass of a material
point is discarded only when an eroded material point is not connected
to any nodes.\\

As can be noticed, in the eigenerosion algorithm an energetic
criterion is adopted. Due to that fact, unrealistic stress
concentration (higher than tensile strength) in quasi-brittle materials, see
\cite{Navas_2017_ES}. To overcome this limitation, the aforementioned
authors proposed the concept of eigensoftening to take in to account
the gradual failure in quasi-brittle materials. The concept is
inspired in the cohesive fracture widely employed in the context of
FEM \cite{Ortiz_1999}. This gradual failure criterion is plotted in
figure \label{fig:Damage-ft-wc}, where a linear decreasing cohesive
law is presented to illustrate the concept here described. In the
picture, the shaded area represents the static fracture energy per
unit of area, $G_F$. As we can see, a cohesive crack appears when the
maximum tensile strength, $f_t$ is reached. Once the opening
displacement $w$ takes the value of the critical crack displacement
$w_c$, a stress-free crack is attained. For intermediate values 
%%%%%%%%%%%
\begin{figure}
  \centering
  \includegraphics[width=0.5\textwidth]{Figures/Damage}
  \caption{Scheme of a linear cohesive law, where the shaded area is
    $G_f$, $f_t$ is the tensile strength, and $w_c$ is the critical
    opening displacement.}
  \label{fig:Damage-ft-wc}
\end{figure}
%%%%%%%%%%%


\begin{equation}
  \label{eq:variation-averaged-strain-energy-density}
  \delta W_{p,\epsilon} = \frac{\partial G_p}{C_{\epsilon}} =
  \frac{1}{m_p} \sum_{x_q^{k+1} \in
  B_{\epsilon}(x_p^{k+1})} m_q \tens{\sigma}_{q,I} \delta \tens{\epsilon}_q
\end{equation}

\begin{equation}
  \label{eq:variation-averaged-strain-energy-density-simpli}
  \delta W_{p,\epsilon} =
  \frac{\delta \tens{\epsilon}_p}{m_p} \sum_{x_q^{k+1} \in
  B_{\epsilon}(x_p^{k+1})} m_q \tens{\sigma}_{q,I} 
\end{equation}

\begin{equation}
  \label{eq:equivalent-critical-stress}
  \delta \tens{\sigma}_{p,c} =
  \frac{1}{m_p} \sum_{x_q^{k+1} \in
  B_{\epsilon}(x_p^{k+1})} m_q \tens{\sigma}_{q,I} 
\end{equation}

%%%%%%%%%%%%%%%%%%%%%%%%%%%%%%%%%%%%%%%%%%%%%%%%%%%%%%%%%%%%%%%%%%%%%%%%%
\section{Cases of study and discussion}
\label{sec:3}

\subsection{Comparison with analytical solution}
\label{sec:3.1}

\subsection{Brazilian test}
\label{sec:3.2}

\begin{figure}
  \centering
  \includegraphics[width=0.5\textwidth]{Figures/Brazilian}
  \caption{Geometry and boundary condition of the Brazilian test.}
  \label{fig:geometry-brazilian-test}
\end{figure}

\subsection{Drop-weight impact test}
\label{sec:3.3}

\begin{figure}
  \centering
  \includegraphics[width=0.8\textwidth]{Figures/Drop_weight}
  \caption{Geometry and boundary condition of the drop-weight impact test.}
  \label{fig:geometry-drop-weight-impact-test}
\end{figure}


%%%%%%%%%%%%%%%%%%%%%%%%%%%%%%%%%%%%%%%%%%%%%%%%%%%%%%%%%%%%%%%%%%%%%%%%%
\section{Conclusions}
\label{sec:6}


\section*{Acknowledgements}
The first author acknowledges the fellowship Agustín de Betancourt 262390106114.

\appendix
%\addappheadtotoc
%\appendixpage
%\renewcommand{\theequation}{\Alph{section}.\arabic{equation}}

\clearpage

\section{Explicit Predictor-Corrector algorithm}
\label{sec:expl-pred-corr}

\begin{algorithm}
  \floatname{algorithm}{Algorithm}
  \renewcommand{\thealgorithm}{}
  \caption{Explicit Predictor-Corrector scheme}
  \begin{algorithmic}[1]
    %%%%%%%%%%%%%%%%%%%%%%%%%%%%%%%%%%%%%%%%%%%%%%%%%%%%%%%%%%%%%%%%%%%%%%%%%%%%%%%%%%%%%% º
    \STATE \textbf{Update mass matrix}:
    \begin{equation*}
      \tens{m}_{I} = N_{Ip}^{k}\ m_p,
    \end{equation*}
    %%%%%%%%%%%%%%%%%%%%%%%%%%%%%%%%%%%%%%%%%%%%%%%%%%%%%%%%%%%%%%%%%%%%%%%%%%%%%%%%%%%%%% 
    \STATE \textbf{Explicit Newmark Predictor}:\\
    \begin{equation*}
      \vec{v}_I^{pred} = \frac{ N_{Ip}^{k} m_p (\vec{v}_p^k + (1 - \gamma)\ \Delta t\ \vec{a}_p^k)}{m_I}
    \end{equation*}
    %%%%%%%%%%%%%%%%%%%%%%%%%%%%%%%%%%%%%%%%%%%%%%%%%%%%%%%%%%%%%%%%%%%%%%%%%%%%%%%%%%%%%% 
    \STATE \textbf{Impose essential boundary conditions}:\\
    At the fixed boundary, set $\vec{v}_{I}^{pred} = 0$. 
    %%%%%%%%%%%%%%%%%%%%%%%%%%%%%%%%%%%%%%%%%%%%%%%%%%%%%%%%%%%%%%%%%%%%%%%%%%%%%%%%%%%%%% 
    % \STATE \textbf{Discard the previous nodal values}.
    %%%%%%%%%%%%%%%%%%%%%%%%%%%%%%%%%%%%%%%%%%%%%%%%%%%%%%%%%%%%%%%%%%%%%%%%%%%%%%%%%%%%%% 
    \STATE \textbf{Deformation tensor increment calculation}.
    \begin{align*}
      &\dot{\tens{\varepsilon}_{p}}^{k+1} = \left[ \vec{v}_{I}^{pred} \otimes
        \Grad{N_{Ip}^{k+1}} \right]^s \\
      &\Delta \tens{\varepsilon}_{p}^{k+1} = \Delta t\ \dot{\tens{\varepsilon}_{p}}^{k+1}
    \end{align*}
    %%%%%%%%%%%%%%%%%%%%%%%%%%%%%%%%%%%%%%%%%%%%%%%%%%%%%%%%%%%%%%%%%%%%%%%%%%%%%%%%%%%%%% 
    \STATE \textbf{Update the density field}:
    \begin{equation*}
      \rho_p^{k+1} = \frac{\rho_p^k}{1 + \mathit{tra}\left[\Delta\tens{\varepsilon}_{p}^{k+1}\right]}.
    \end{equation*}
    %%%%%%%%%%%%%%%%%%%%%%%%%%%%%%%%%%%%%%%%%%%%%%%%%%%%%%%%%%%%%%%%%%%%%%%%%%%%%%%%%%%%%% 
    \STATE \textbf{Compute damage parameter}:
    %%%%%%%%%%%%%%%%%%%%%%%%%%%%%%%%%%%%%%%%%%%%%%%%%%%%%%%%%%%%%%%%%%%%%%%%%%%%%%%%%%%%%% 
    \STATE \textbf{Balance of forces calculation}:\\
    Calculate the total grid nodal force $\vec{f}_{I}^{k+1} =
    (1-\chi)\vec{f}_{I}^{int,k+1} + \vec{f}_{I}^{ext,k+1}$ evaluating
    \eqref{eq:nodal_internal_forces} and
    \eqref{eq:nodal_external_forces} in the time step $k+1$.
    In the grid node $I$ is fixed in one of the spatial dimensions, set it to
    zero to make the grid nodal acceleration zero in that direction.\\
    %%%%%%%%%%%%%%%%%%%%%%%%%%%%%%%%%%%%%%%%%%%%%%%%%%%%%%%%%%%%%%%%%%%%%%%%%%%%%%%%%%%%%% 
    \STATE \textbf{Explicit Newmark Corrector}:
    \begin{equation*}
      \vec{v}_{I}^{k+1} = \vec{v}_{I}^{pred} + \gamma\ \Delta t\ \frac{\vec{f}_{I}^{k+1}}{\tens{m}_I^{k+1}}  
    \end{equation*}
    %%%%%%%%%%%%%%%%%%%%%%%%%%%%%%%%%%%%%%%%%%%%%%%%%%%%%%%%%%%%%%%%%%%%%%%%%%%%%%%%%%%%%%
    \STATE \textbf{Update particles lagrangian quantities}:
    \begin{align*}
      &\vec{a}_p^{k+1} = \frac{N_{Ip}^k\vec{f}_{I}^{k}}{\tens{m}_I^k}\\
      &\vec{v}_p^{k+1} = \vec{v}_p^n + \Delta t\
        \frac{N_{Ip}^k\
        \vec{f}_{I}^{k}}{\tens{m}_I^k}\\
      &\vec{x}_p^{k+1} = \vec{x}_p^n + \Delta t\
         N_{Ip}^k\ \vec{v}_{I}^{k} +
        \frac{1}{2}\Delta t^2\ \frac{N_{Ip}^k\
        \vec{f}_{I}^{k}}{\tens{m}_I^k}
    \end{align*}
    %%%%%%%%%%%%%%%%%%%%%%%%%%%%%%%%%%%%%%%%%%%%%%%%%%%%%%%%%%%%%%%%%%%%%%%%%%%%%%%%%%%%%% 
    \STATE \textbf{Reset nodal values}
  \end{algorithmic}
\end{algorithm} 


\section{Eigensoftening Algorithm}
\label{sec:eigens-algor-1}

\begin{algorithm}\caption{Compute damage parameter $\chi_P^{k+1}$}\label{alg-eigens}
  \begin{algorithmic}
    \REQUIRE Particle status\\
    Number of particles: $N_p$\\
    $\epsilon$-neighbourhood of each particle $p$ : $B_{\epsilon,p}$\\
    \REQUIRE Material data\\
    Tensile strength: $f_{t,p}$\\
    Bandwidth of the cohesive fracture: $h_{\epsilon,p}$ \\
    Critical opening displacement: $w_c$\\ 
    \ENSURE Return damage parameter $\chi := \{\chi_p\}$
    \STATE $\chi_p \leftarrow \chi_p^{k}$
    \FOR{$p$ to $N_p$}
    \IF{$\chi_p = 0$ \AND $\epsilon_{f,p} = 0$}
    \FOR{$q \in B_{\epsilon,p}$}
    \IF{$\chi_q < 1$}    
    \STATE $\sum m_p\sigma_{p,I} \leftarrow \sum m_p\sigma_{p,I} + m_q\sigma_{q,I}$
    \ENDIF    
    \STATE $m_p \leftarrow m_p + m_q$
    \ENDFOR
    \STATE $\sigma_{p,\epsilon} \leftarrow \frac{1}{m_p} \sum m_p\sigma_{p,I}$
    \IF{$\sigma_{p,\epsilon} > f_{t,p}$}
    \STATE $\epsilon_{f,p} = \epsilon_{I,p}$   
    \ENDIF        
    \ELSE[$\chi_p \neq 1$ \AND $\epsilon_{f,p} > 0$]
    \STATE $\chi_p^{k+1} \leftarrow \min\Big \{1 , \max \{\chi_p^{k},
    \frac{(\epsilon_{I,p}- \epsilon_{f,p})\ h_{\epsilon,p}}{w_c} \} \Big \}$    
    \ENDIF    
    \ENDFOR
  \end{algorithmic}
\end{algorithm}


% name your BibTeX data base
\bibliography{Biblio/Biblio} 

\end{document}