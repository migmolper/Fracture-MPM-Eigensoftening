\documentclass[12pt]{article}
\usepackage{geometry} % see geometry.pdf on how to lay out the page. There's lots.
\geometry{a4paper} % or letter or a5paper or ... etc
\usepackage{graphicx}
\usepackage{cite}
\usepackage{color}
\usepackage{bm}
\usepackage{amsmath}
\usepackage{textcomp}

% \geometry{landscape} % rotated page geometry

\title{Response to the reviewers of the manuscript ``LME-MPM applied to quasi-brittle fracture."}
\author{Miguel Molinos, Pedro Navas, Diego Manzanal and Manuel Pastor\\}
\date{}

%\date{} % delete this line to display the current date

%%% BEGIN DOCUMENT
\begin{document}

\maketitle
%\tableofcontents

We are grateful to the reviewers for taking their time  to review our work. Their comments have helped us to improve the paper. Changes to the original manuscript are given in  color \textcolor{red}{red} for corrections and  \textcolor{blue}{blue} for modified text in the revised version.  A detailed response to both reviewers is given below.

\section*{Response to Editor}

 \begin{enumerate}

\item \textit{Please include a nomenclature in alphabetic order: Latin, Greek, Acronyms.}\\

Done.
 
\item \textit{Reduce the list of references to the essential papers only..}\\

Done.
  
\item \textit{Please avoid acronyms as LME-MPM in the tittle and expand the abbreviations.}\\

Done.

 \end{enumerate}

\section*{Response to Reviewer \#1}
{\it
I am aware of several approaches to modeling fracture in MPM from explicit methods (CRAMP) to methods based on material softening. This paper provides yet another option. I am not convinced of its benefits, but it can still be published with revisions (and with cleaning up of some of the English wording).
}
\\
\\
The reviewer's positive commentaries are greatly appreciated.
\\

\textit{The authors rely on a variant of MPM called LME-MPM. I was not familiar with this option before. This paper might need more explanation, but maybe the references are enough.}\\

The LME-MPM is explained in more detail in an earlier article by the authors. In this paper, only several notions are provided. This notions have been extended in order to improve the understanding of it
\\

\textit{Here are some specific comments:}

 \begin{enumerate}
 
 \item \textit{"proofs" should be "proves".}\\
 
 Fixed.

 \item \textit{"numerical artifacts have been proposed" People don't intentionally create artifacts. I think authors meant something different. This paper needs some English editing, but I will not comment on them all.}\\
 
The correct word is "artifices" instead of artifacts. As suggested by the reviewer, a proofreading will be done to improve the English of the manuscript.
 
 \item \textit{Page 4: Equation (1) (from standard MPM) must sum over particles, but that is not stated. Perhaps they are using "repeated indices" summed, but that is not stated either. Also, MPM codes typically do not track particle acceleration. How is that terms tracked here? (actually I see in Eq, (3), but that is not common in MPM, instead, that extrapolation is used to update particle position and then discarded).}\\
 
As the reviewer noticed, the Einstein summation convention for repeated indices has been adopted. This will be clarified in the revised version of the manuscript. Concerning the use of the particle acceleration in the algorithm, this a key part of the Newmark predictor-corrector (NPC) proposed by the authors in Molinos {\it et al.}  \cite{Molinos_2020} (not yet published). This algorithm detailed in the Appendix A. Additionally,  since the algorithm proposed in the aforementioned publication was adapted for the MPM from the proposed by Navas {\it et al.} \cite{Navas_2018a} (2018). More details of the good performance of the NPC algorithm in the dynamic region can be found in the aforementioned paper. To clarify this issue, an additional reference to this work will be incorporated to the revised version of the manuscript.

\item \textit{Page 7: "reinforces" should be "reinforced" (in FRC)}\\

Fixed.

\item \textit{Section 2.3: Fracture modelling approach. The core of this paper appears to be "isotropic softening" where particles degrade the same in all directions. Although this is common approach in numerical modeling, I am skeptical it can really extend fracture mechanics. For example, how does this modeling handle mode I (failure by tensile stress) from mode II (failure by shear stress). An alternative is to use anisotropic softening and that approach has been implemented in MPM: J. A. Nairn, "Direct Comparison of Anisotropic Damage Mechanics to Fracture Mechanics of Explicit Cracks," Eng. Fract. Mech., 203, 197-207 (2018) and papers cited therein (citing such papers would also help promote impact factor of Engineering Fracture Mechanics).}\\
 
Authors appreciate the suggestion of the reviewer to mention interesting affine works. The state of art of the manuscript will be revised to include this research and remove the non necessary references. Answering reviewer\textquotesingle s question, the eigensoftening technique is able to handle with both modes I and II. Observe that equation (14) is formulated in terms of $\sigma_{I}$. This is the first component of the stress tensor in the maximum principal stress. Therefore, fracture will appear following the $\sigma_I$ trajectories. For instance, it it is well known that a pure shear stress state ($\tau$) can be expressed as a composition of two stress states, one traction ($\sigma_{I} = \tau$) and other compression ($\sigma_{II} = -  \tau$) with principal axes rotated 45$^o$. Additionally, as  Navas {\it et al.}  \cite{Navas_2018b} proved by means of qualitative and quantitative comparisons, the eigensofteing technique is able to deal with mixed modes even with non isotropic materials as the FRC.
 
\item \textit{General: What shape functions are used or what are the $N_I$? It was not clear to me. The nomenclatture "N" is common used in MPM and finite element literature to refer to cell shape functions. MPM integrates these of particle volume to get GIMP shape functions.}\\

$N_I$ is used to define the value of the shape function for a given node $I$ in any position of the space $x$. The shape functions here described (LME) has global support, consequently in this case it does not refer to \textit{cell} shape functions. This concept will be better explained int the  manuscript to avoid misunderstood.
  
\item \textit{Page 12: "four particles per element and occupying the Gauss quadrature positions" MPM does not numerically integrate particles and they move while the grid does not. Why start at Gaussian quadrature positions? I have never seen that in MPM. The usual discretization is to use equally spaced material points (or at quarter points of the cells).}\\

In MPM, the user is able to choose the initial position of the particles. Therefore, we choose the Gaussian quadrature positions to ensure good integration properties at least in the first time steps. Of course, in a general scenario, the user has not control over the position of the material points in the following time steps. 
   
\item \textit{Page 16: "Figure 7 shows the comparison of mixed-mode fracture"
Isotropic failure methods can be used on mixed-mode failure and might even give results that look reasonable. But mixed mode results should depend on separate stresses to initiation tensile failure or shear failure and different amounts critical energy for mode I and Mode II. For example, if this work uses a single toughness, all failure will result in the same energy regardless of ratio of mode I to mode II. That response does not agree well with experimental results (e.g., mixed-mode delamination or adhesive failure).}\\

As the reviewer argues, to simulate complex fracture patterns as delamination, the model should consider an anisotropic behaviour. Although, contrary to the reviewer\textquotesingle s opinion, the eigensofening is capable to capture mixed-mode failure. In previous research, Navas {\it et al.}  \cite{Navas_2018b} demonstrate against experimental results that the eigensoftening approach is able to capture properly the crack pattern and other experimental measurements. 

However, we agree with the reviewer that in order to capture complex behaviours such as delamination, the damage model for concrete must consider the anisotropy of the material in the formulation. Therefore, the authors agree with the reviewer that the use of an anisotropic damage model would be an important step forward for the proposed model. This pitfall will be clarified and the proposal of an anisotropic damage model will be included in the conclusions as a future research line of the authors.

\item \textit{Page 19: "Although the traditional MPM implementation allows collision of two bodies, inaccurate energy conservation is obtained without the implementation of a conservative contact algorithm to reproduce the hammer impact in the beam." Well known and many papers are available that do contact in MPM (including many improvements over the two cited in Refs 51 and 52). It is one of the common reasons for using MPM instead of finite element modeling.}\\

As the reviewer remarks, better results would be obtained if a better contact algorithm is implemented. In addition, the implementation of a proper contact algorithm would help to model the supports of the concrete beam in a more realistic manner (Section 3.3). 

\item \textit{Page 19: "opposite to the uGIMP shape function, LME allows unstructured grids" Actually, uGIMP has no problem handling a rectilinear grid with arbitrarily sized elements. It could even handle unstructured grids with extra work, but I have never seen it done.}\\

It is true that uGIMP has no problem to handle a rectilinear grid with arbitrarily sized elements since the construction of this shape functions is based on generating 2D or 3D shape functions by multiplicative construction of a closed 1D expression. Unfortunately, in the absence of a regular grid, construction of the weighting functions is only achieved at considerable effort and computational cost as Bardenhagen and Kober \cite{Bardenhagen_2004} (2004) states. This thought will be included in the revised manuscript. 

\item \textit{General: I did not find any mention of handling crack contact either during fracture (such as pure mode II fracture) on when a crack forms and then comes back into contact when forces later change direction. Is it possible in this methods or is it a limitation of the approach (such limitations should be listed). The mentioned CRAMP algorithm fully handles contact using results similar to MPM contact methods. I recall the anisotropic softening methods handle crack contact as well. The challenge is that when a crack in progress comes into contact, it should respond with a stiffness similar to the initial material and not some isotropically reduced stiffness. Can this approach distinguish tension and compression? Real materials have very different failure mode.}\\

The authors appreciate the constructive discussion of the reviewer. Extending the eigensoftening algorithm to anisotropic softening seems to be an interesting research line in the future. It will be added to the conclusions. Handling crack contact is a current pitfall of the algorithm and will be properly listed in the corrected version of the manuscript. Furthermore, in Pandolfi and Ortiz (2012) \cite{Ortiz_Pandolfi_2012} (Section 3.2) two alternatives to overcome this limitation were briefly discussed. Since the examples proposed in the present research does not involves contact constrain, those improvements were not the focus of our research. This pitfall will be included in the description of the algorithm. 

\item \textit{uGIMP: uGIMP can be based on linear or spline grid shape functions. The later can eliminate all oscillations seen in linear uGIMP. The improvements are most dramatic in bending problems such as example 2 and 3 here. Did authors try them?}\\

The authors appreciate the interesting proposal of the reviewer. We were aware of the work proposed by Steffen {\it et al.} \cite{Steffen_2008} with the B-splines. But we were not aware of the possibility of using spline grid shape functions in the uGIMP. However, since the aim of this manuscript is the discussion of the capabilities of a new fracture algorithm for the MPM, we only considered the linear uGIMP shape functions and the LME approximants to establish a comparison. Nevertheless, the discussion proposed by the reviewer seems to be interesting for a work focused on the computational aspect of the MPM.

 \end{enumerate}

\hspace{5mm}



\section*{Response to Reviewer \#2}

\textit {The manuscript describes an approach to tackle fracture process in brittle materials. Using the material point method as discretisation and the eigendeformation concept (in the version devised by one of the authors in his previous contributions), authors propose an efficient algorithm that is illustrated by a few examples of application. The paper is well written, the literature adequately acknowledged. }\\



\textit{A few comments.}
 \begin{enumerate}
 
\item \textit{If you can reduce the references to the necessary ones, it would be great.}\\

Done.

\item \textit{The introduction to the case of study in section 3 is not necessary, you can shorten it.}\\

We agree with the reviewer\textquotesingle s opinion, unnecessary references  and explanations will be removed. 

\item \textit{Images look as pixels. Is this an artefact of the approach or figures can be obtained in a smoother shade of colors ?}\\

We are printing the evolution of stress and other magnitudes in the material points. Therefore, each "pixel" represents a material point. We will add an explanatory note in the legend of the figures to avoid misunderstandings.

\item \textit{Units of measure are not mathematical symbols, they MUST not be written in italic (they are wrong in Table 1 and in the text and correct in Table 2).}\\

Done.

\item \textit{What is the rationale behind the choice of the material parameters in Example 3.1? The units seem to be dimensionless, so please remove the dimensions, they are unnatural.}\\

The material parameters are the same as those used by Pandolfi and Ortiz (2012) \cite{Ortiz_Pandolfi_2012} for the eigenerosion approach. Dimensions will be removed.

\item \textit{Can you explain why there is a wrong time history in the simulation of the drop-weight test?
}\\

The reason lies on the fact that the real hammer, used in the experimental tests, is not reproduced exactly on the simulations in order to avoid to reproduce an enormous hammer compared to the beam. In instead of it, we simulated a hammer with two different materials. The "head" of the numerical hammer has the same properties as the one used in the experiments, the "body" of the numerical hammer has different density and elastic modulus. This approach makes it possible to considerably reduce the length of the numerical hammer and, therefore, the computational cost of the simulation. The elastic modulus and density of the second material were calibrated to reproduce accurately the impact, unfortunately after it, the numerical hammer does not behave exactly as the one used in the experiments. 

\end{enumerate}


\bibliographystyle{unsrt}
\begin{thebibliography}{10}

\bibitem{Bardenhagen_2004}
Bardenhagen, S. and Kober, E. M.
\newblock {The generalized interpolation material point method}.
\newblock {Computer Modeling in Engineering and Sciences},  2004

\bibitem{Steffen_2008}
Steffen, M. and Wallstedt, Philip and Guilkey, James and Kirby, Robert and Berzins, Martin.
\newblock {Examination and analysis of implementation choices within the Material Point Method (MPM)}.
\newblock {CMES - Computer Modeling in Engineering and Sciences}, 2008

\bibitem{Molinos_2020}
Molinos, M. and Navas, P. and Pastor, M. and Mart\'in Stickle, M.
\newblock {On the dynamic assessment of the Local-Maximum Entropy Material Point Method through an Explicit Predictor-Corrector Scheme}.
\newblock {Computer Methods in Applied Mechanics and Engineering},  Under review,

\bibitem{Navas_2018a}
Navas, P. and L{\'{o}}pez-Querol, S and Yu, R. C. and Pastor, M.
\newblock {Optimal transportation meshfree method in geotechnical engineering problems under large deformation regime}.
\newblock {International Journal for Numerical Methods in Engineering},  2018.

\bibitem{Navas_2018b}
Navas, P. and Yu, R. and Ruiz, G.
\newblock {Meshfree modeling of the dynamic mixed-mode fracture in FRC through an eigensoftening approach}.
\newblock {Engineering Structures},  2018.

\bibitem{Ortiz_Pandolfi_2012}
Pandolfi, A. and Ortiz, M.
\newblock {An eigenerosion approach to brittle fracture}.
\newblock {International Journal for Numerical Methods in Engineering},  2012.

\end{thebibliography}

\end{document}





